Uma Maquina de Vetores de Suporte é um método de aprendizado de máquina supervisionado. Seu modelo é binário de forma que após o treinamento é possível classificar um exemplo entre duas classes. Seu treinamento consiste em iterar diversas vezes sobre todo o conjunto de treinamento realizando operações complexas no processo. Para classificação ela separa um subconjunto e realiza as mesmas operações sobre elas para chegar em uma estimativa. Esse processo é muito custoso já que as operações são complexas e um conjunto de dados precisa ser grande para render boa precisão na classificação. Por isso é um método interessante para ser paralelizado em GPU já que é feita a mesma operação em milhares de dados sem dependência entre eles.

Foi criado um programa capaz de rodar uma Maquina de Vetores de Suporte de forma sequencial ou paralela. O algoritmo escolhido para implementar foi KAA, por ser um algoritmo que resolve os problemas da maquina de vetores de suporte pela força bruta, tem um alto potencial de ganho com paralelização em GPU. É feita uma analise do desempenho do programa alterando diversos fatores que modificam a execução do programa tanto pelo lado da MVS quanto pela GPU, o objetivo é encontrar o maior ganho de velocidade da GPU e o maior ganho de precisão da MVS.